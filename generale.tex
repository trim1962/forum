\chapter{TeX e LaTeX}
\label{chap:texelatex}
\section{Generale}
\label{sec:Texgenerale}
\subsection{Gestione discussione}

\begin{domanda}{domanda e risposta}{claudio}{Tex e Latex}
come fare per gestire dei blocchi come questo
Domanda \verb!<titolo della domanda>! Autore \verb!< Nome autore >! \verb!\index{nome}!
blocco della domanda bla bla bla bla bll
Risposta : Autore risposta \verb!< Nome autore>\index{Nome}!
blocco della risposta bla bla bla
ciao
claudio 
\end{domanda}
\begin{risposta}{cfiandra}
	\begin{lstlisting}{}
\documentclass{article}
\usepackage[T1]{fontenc}
\usepackage{textcomp}
\usepackage[italian]{babel}
\usepackage{tcolorbox}
	\newenvironment{domanda}[3]{
\begin{tcolorbox}[width=\textwidth,colback=red!5,
colframe=red!75!black,title=#1]%
Autore: #2\\
Categoria: #3
\tcblower
}{\end{tcolorbox}\bigskip}
\newenvironment{risposta}[1]{
\begin{tcolorbox}
[width=\textwidth,colback=green!5,
colframe=green!60!black,title=#1]} %
{\end{tcolorbox}\bigskip} 
\begin{document}
\begin{domanda}{domanda e risposta}{claudio}{\TeX{}e\LaTeX{}}
come fare per gestire dei blocchi come questo
Domanda \verb!<titolo della domanda>! 
Autore \verb!< Nome autore >! \verb!\index{nome}!
blocco della domanda bla bla bla bla bll
Risposta : Autore risposta 
\verb!< Nome autore>\index{Nome}!
blocco della risposta bla bla bla
ciao
claudio
\end{domanda}
\begin{risposta}{OldClaudio}
Vuoi mantenere domanda e risposta unite, 
senza salti di pagina in mezzo? O che cosa intendi 
per ``gestire''?
\end{risposta}
\begin{risposta}{claudio}
in pratica vorrei fare una raccolta di domande e risposte 
di questo forum quindi non ho idea della domanda e della 
risposta e della loro strutturazione impossibile da gestire?
ciao
claudio
\begin{risposta}{OldClaudio}
Impossibile non saprei, a difficile senz'altro, molto difficile; 
pensa a un topic qualsiasi; uno fa una domanda e riceve via via 
dieci risposte delle quali una solleva un polverone,su cui si 
scatenano altre risposte che spesso non hanno nulla a che vedere con la prima
domanda; oppure che generano alte domande nello stesso filone. 
Le risposte cominciano ad intrecciarsi e non si capisce pi\'u 
chi stia rispondendo a chi, sebbene la lettura del filone di 
discussione sia spesso molto interessante per la variet\'a  
dei punti di vista e per la dinamica che nasce a chi, sebbene la 
lettura del filone di discussione sia spesso molto interessante 
per la variet\'a dei punti di vista e per la dinamica che nasce	
da questi dialoghi ``in piazza'', ``nel foro (romano)''
\end{risposta}
\begin{risposta}{claudio}
non tutte le risposte sarebbero riprese ma solo quella che risolve il problema.
ciao 
\end{risposta}
\end{document}
\end{lstlisting}

\end{risposta}